\documentclass{article}
\usepackage[utf8]{inputenc}
\usepackage{amsmath, amssymb}

\usepackage{biblatex}
\addbibresource{references.bib}


\newcommand{\pderiv}[2]{\frac{\partial #1}{\partial #2}}
\newcommand{\npderiv}[3]{\frac{\partial^#1 #2}{\partial #3^#1}}
\newcommand{\pderivconst}[3]{\left(\pderiv{#1}{#2}\right)_{#3}}


\title{Implementation of Multi-component Soave-Redlich-Kwong Equation of State in Cantera}
\author{Cory Kinney \\ University of Central Florida}
\date{May 2023}

\begin{document}
\maketitle


\section{Overview}
In addition to the ideal gas equation of state (EoS), \textsc{Cantera} \cite{cantera} currently has implementations of the Redlich-Kwong \cite{redlichkwong} and Peng-Robinson \cite{peng1976new} real gas equations of state. This paper outlines the approach taken to implement the Soave modification \cite{soave1972equilibrium} of the Redlich-Kwong EoS. 

 

\section{Soave-Redlich-Kwong EoS}
The Soave-Redlich-Kwong (SRK) EoS is given by
\begin{equation}
p = \frac{RT}{\nu-b}-\frac{a\alpha}{\nu(\nu+b)}\label{eq:SRK}
\end{equation}
where $p$ is pressure, $R$ is the universal gas constant, $T$ is temperature, and $\nu$ is the molar volume. The species-specific Van der Waals attraction parameter $a$ and the the repulsive, volume correction parameter $b$ are calculated as
\begin{equation}
    a = \frac{\Omega_a R^2 T_c^2}{p_c}\label{eq:a}
\end{equation}
\begin{equation}
    b = \frac{\Omega_b R T_c}{p_c}\label{eq:b}
\end{equation}
where the constants $\Omega_a$ and $\Omega_b$ are defined as
\begin{equation}
    \Omega_a = \left[9(2^{1/3}-1)\right]^{-1} \approx 0.42748
\end{equation}
\begin{equation}
    \Omega_b = \frac{2^{1/3}-1}{3} \approx 0.08664
\end{equation}
The temperature dependent interaction parameter $\alpha$ is given as
\begin{equation}
    \alpha(T) = \left[1 + \kappa\left(1 - \sqrt{\frac{T}{T_c}}\right)\right]^2 \label{eq:alpha}
\end{equation}
where $\kappa$ is given by
\begin{equation}
    \kappa = 0.480 + 1.574\omega - 0.176\omega^2
\end{equation}
which is a function of the acentric factor ($\omega$) of the species.


\subsection{Multi-component Soave-Redlich-Kwong EoS}
The pure fluid SRK EoS can be generalized for a multi-component mixture as
\begin{equation}
    p = \frac{RT}{\nu-b_{mix}}-\frac{(a\alpha)_{mix}}{\nu(\nu+b_{mix})}\label{eq:multicomponent}
\end{equation}
The mixture averaged properties are defined as 
\begin{equation}
    a_{mix} = \sum_i \sum_j X_i X_j a_{ij}
\end{equation}
\begin{equation}
    b_{mix} = \sum_i X_i b_i
\end{equation}
\begin{equation}
    (a\alpha)_{mix} = \sum_i \sum_j X_i X_j (a\alpha)_{ij}\label{eq:aalpha_mix}
\end{equation}
The default species interaction parameters $a_{ij}$ and $(a\alpha)_{ij}$ are defined by the geometric average of the pure-species parameters
\begin{equation}
    a_{ij} = \sqrt{a_i a_j}
\end{equation}
\begin{equation}
    (a\alpha)_{ij} = \sqrt{(a\alpha)_i(a\alpha)_j}\label{eq:aalpha_ij}
\end{equation}


\subsection{Critical properties}
Critical properties can be calculated from the species parameters, $a$ and $b$, and the EoS-specific constants.
Dividing \eqref{eq:a} by \eqref{eq:b} gives
\begin{equation*}
    \frac{a}{b} = \frac{\Omega_aRT_c}{\Omega_b}
\end{equation*}
solving for critical temperature yields
\begin{equation}
    T_c = \frac{a\Omega_b}{b\Omega_a R}\label{eq:Tc}
\end{equation}
Rearranging \eqref{eq:b} to solve for critical pressure
\begin{equation*}
    p_c = \frac{\Omega_bRT_c}{b}
\end{equation*}
and then substituting \eqref{eq:Tc} yields
\begin{equation}
    p_c = \frac{a\Omega_b^2}{\Omega_ab^2}\label{eq:PC}
\end{equation}
The critical properties for the multi-component case can also be calculated from the mixtured averaged properties using 
\begin{equation}
    T_{c,mix} = \frac{a_{mix}\Omega_b}{b_{mix}\Omega_a R}\label{eq:Tc_mix}
\end{equation}
\begin{equation}
    p_{c,mix} = \frac{a_{mix}\Omega_b^2}{\Omega_a{b_{mix}}^2}\label{eq:PC_mix}
\end{equation}


\subsection{Cubic Form}
The cubic form in terms of molar volume can be obtained by multiplying both sides of \eqref{eq:SRK} by the denominators of the right hand side which gives
\begin{equation*}
    p\nu(\nu-b)(\nu+b) = RT\nu(\nu+b) - a\alpha(\nu-b)
\end{equation*}
expanding and rearranging gives
\begin{equation*}
    p\nu(\nu^2-b^2) = RT(\nu^2+b\nu)-a\alpha(\nu-b)
\end{equation*}
\begin{equation*}
    p\nu^3-b^2p\nu-RT\nu^2-bRT\nu+a\alpha\nu-a\alpha b=0
\end{equation*}
finally, grouping like terms and dividing by pressure yields the cubic form
\begin{equation}
    \nu^3-\frac{RT}{p}\nu^2+\left(\frac{a\alpha - bRT}{p}-b^2\right)\nu - \frac{a\alpha b}{p}=0\label{eq:cubic}
\end{equation}
The cubic form can also be written in terms of the compressibility factor ($Z$) which is defined as
\begin{equation}
    Z = \frac{P\nu}{RT}
\end{equation}
Therefore, substituting the corresponding definition of molar volume in terms of compressibility factor
\begin{equation}
    \nu = \frac{ZRT}{P}\label{nuZ}
\end{equation} 
into \eqref{eq:cubic} gives
\begin{equation*}
    \left(\frac{RT}{P}\right)^3Z^3 - \left(\frac{RT}{P}\right)^3Z^2 + \frac{RT}{p}\left(\frac{a\alpha - bRT}{p}-b^2\right) Z - \frac{a\alpha b}{p} = 0
\end{equation*}
dividing through by the first coefficient yields
\begin{equation*}
    Z^3 - Z^2 + \left[\frac{a\alpha p}{R^2T^2} - \frac{bp}{RT} - \left(\frac{bp}{RT}\right)^2\right] Z - \frac{a\alpha bp^2}{R^3T^3} = 0
\end{equation*}
which can finally be written as
\begin{equation}
    Z^3 - Z^2 + (A - B - B^2)Z - AB = 0\label{Zcubic}
\end{equation}
where the terms are $A$ and $B$ are given by
\begin{equation}
    A = \frac{a\alpha p}{R^2 T^2}
\end{equation}
\begin{equation}
    B = \frac{bp}{RT}
\end{equation}
or for the multi-component SRK
\begin{equation}
    A_{mix} = \frac{(a\alpha)_{mix} p}{R^2 T^2} \label{eq:A}
\end{equation}
\begin{equation}
    B_{mix} = \frac{b_{mix} p}{RT} \label{eq:B}
\end{equation}


\section{Thermodynamic Properties}
\subsection{Helmholtz Energy Departure Function}

\begin{equation}
    \pderiv{a}{\nu}\biggr\rvert_{n_k,T} = -p
\end{equation}

\begin{equation}
    \int_{a^\circ}^a da = -\int_{\infty}^{\nu} p d\nu
\end{equation}

\begin{equation}
    a - a^\circ = \int_{\infty}^{\nu^\circ}pd\nu - \int_{\infty}^{\nu}pd\nu
\end{equation}

And then a couple steps later

\begin{equation}
    a - a^\circ = RT \ln\left(\frac{\nu^\circ}{\nu - b}\right) + \frac{a\alpha}{b}\ln\left(\frac{\nu}{\nu+b}\right)
\end{equation}

\subsection{Chemical Potential}
The chemical potential of a species $i$ is given by
\begin{equation}
    \mu_i = \pderivconst{A}{n_i}{T,V,n_{j\neq i}}
\end{equation}

\section{Derivatives}
\subsection{Pressure}
The pressure derivatives of the multi-component SRK equation of state \eqref{eq:multicomponent} with respect to $\nu$ and $T$ are given by
\begin{equation}
    \pderivconst{p}{\nu}{T} = -\frac{RT}{(\nu-b_{mix})^2} + (a\alpha)_{mix}\frac{2\nu+b_{mix}}{[\nu(\nu+b_{mix})]^2} \label{dpdv}
\end{equation}
\begin{equation}
    \pderivconst{p}{T}{\nu} = \frac{R}{\nu-b_{mix}} - \frac{1}{\nu(\nu+b_{mix})}\frac{\partial (a\alpha)_{mix}}{\partial T} \label{dpdT}
\end{equation}


\subsection{Molar volume}
\subsubsection{Isothermal Compressibility}
Isothermal compressibility is defined as 
\begin{equation}
    \beta_T = -\frac{1}{\nu} \pderivconst{\nu}{p}{T}
\end{equation}
Since the inverse of this derivative is much more easily calculated, we can apply the inverse function theorem under the assumption that the Jacobian determinant is nonzero
\begin{equation*}
    \pderivconst{\nu}{p}{T} = \pderivconst{p}{\nu}{T}^{-1}
\end{equation*}
Therefore, the isothermal compressibility takes the final form
\begin{equation}
    \beta_T = -\left[\nu \pderivconst{p}{\nu}{T}\right]^{-1} \label{betaT}
\end{equation}
from the derivative of pressure with respect to molar volume \eqref{dpdv}.


\subsubsection{Thermal Expansion Coefficient}
The volumetric thermal expansion coefficient is defined as
\begin{equation}
    \alpha_V = \frac{1}{\nu} \pderivconst{\nu}{T}{p}
\end{equation}
Applying the triple product rule
\begin{equation*}
    \pderivconst{\nu}{T}{p} \pderivconst{p}{\nu}{T} \pderivconst{T}{p}{\nu} = -1 
\end{equation*}
Once again applying the inverse function theorem, and then solving for the derivative of molar volume with respect to temperature yields
\begin{equation*}
    \pderivconst{\nu}{T}{p} = -\frac{\pderivconst{p}{T}{\nu}}{\pderivconst{p}{\nu}{T}}
\end{equation*}
Therefore, the thermal expansion coefficient takes the final form
\begin{equation}
    \alpha_V = -\frac{1}{\nu} \frac{\pderivconst{p}{T}{\nu}}{\pderivconst{p}{\nu}{T}} \label{alphaV}
\end{equation}
from the derivatives of pressure with respect to temperature \eqref{dpdT} and molar volume \eqref{dpdv}. 


\subsection{Mixing Expressions}
Of the mixture averaged parameters, only $(a\alpha)_{mix}$ is temperature-dependent. The derivative of \eqref{eq:aalpha_mix} with respect to 
temperature is simply the sum of the derivatives of all of the species interaction parameters 
\begin{equation}
    \pderiv{(a\alpha)_{mix}}{T} = \sum_i \sum_j X_i X_j \pderiv{(a\alpha)_{ij}}{T}
\end{equation}
Therefore, taking the derivative of \eqref{eq:aalpha_ij} with respect to temperature
\begin{equation*}
    \pderiv{(a\alpha)_{ij}}{T} = \frac{1}{2\sqrt{(a\alpha)_i(a\alpha)_j}} \left[ \pderiv{(a\alpha)_i}{T} (a\alpha)_j + (a\alpha)_i \pderiv{(a\alpha)_j}{T}\right]
\end{equation*}
which can be written
\begin{equation*}
    \pderiv{(a\alpha)_{ij}}{T} = \frac{\sqrt{(a\alpha)_i(a\alpha)_j}}{2} \left[ \frac{1}{(a\alpha)_i}\pderiv{(a\alpha)_i}{T} + \frac{1}{(a\alpha)_j}\pderiv{(a\alpha)_j}{T}\right]
\end{equation*}
and after substituting \eqref{eq:aalpha_ij} yields
\begin{equation}
    \pderiv{(a\alpha)_{ij}}{T} = \frac{(a\alpha)_{ij}}{2} \left[ \frac{1}{(a\alpha)_i}\pderiv{(a\alpha)_i}{T} + \frac{1}{(a\alpha)_j}\pderiv{(a\alpha)_j}{T}\right] \label{eq:daalpha_ij}
\end{equation}
Taking the derivative with respect to temperature a second times goes as follows
\begin{equation*}
\begin{split}
    \npderiv{2}{(a\alpha)_{ij}}{T} = \frac{1}{2} \pderiv{(a\alpha)_{ij}}{T} \left[ \frac{1}{(a\alpha)_i}
    \pderiv{(a\alpha)_i}{T} + \frac{1}{(a\alpha)_j}\pderiv{(a\alpha)_j}{T}\right] \\ + \frac{(a\alpha)_{ij}}{2} 
    \biggl[ \frac{1}{(a\alpha)_i}\npderiv{2}{(a\alpha)_i}{T}-\frac{1}{(a\alpha)_i^2}\pderiv{(a\alpha)_i}{T}^2 + \\
    \frac{1}{(a\alpha)_j}\npderiv{2}{(a\alpha)_j}{T}-\frac{1}{(a\alpha)_j^2}\pderiv{(a\alpha)_j}{T}^2 \biggr]
\end{split}
\end{equation*}
The first term can be simplified using \eqref{eq:daalpha_ij} giving
\begin{equation}
\begin{split}
    \npderiv{2}{(a\alpha)_{ij}}{T} = \frac{1}{(a\alpha)_{ij}} \pderiv{(a\alpha)_{ij}}{T}^2 + \frac{(a\alpha)_{ij}}{2} 
    \biggl[ \frac{1}{(a\alpha)_i}\npderiv{2}{(a\alpha)_i}{T}-\frac{1}{(a\alpha)_i^2}\pderiv{(a\alpha)_i}{T}^2 + \\
    \frac{1}{(a\alpha)_j}\npderiv{2}{(a\alpha)_j}{T}-\frac{1}{(a\alpha)_j^2}\pderiv{(a\alpha)_j}{T}^2 \biggr]
\end{split}
\end{equation}

\subsubsection{Species Parameters}
The first and second derivatives of the species parameters are needed for the mixture derivatives. Since $a$ is not temperature-dependent
\begin{equation}
    \pderiv{(a\alpha)_i}{T} = a_i \pderiv{\alpha_i}{T}
\end{equation}
and
\begin{equation}
    \npderiv{2}{(a\alpha)_i}{T} = a_i \npderiv{2}{\alpha_i}{T}
\end{equation}
Taking the derivative of \eqref{eq:alpha} with respect to temperature
\begin{align*}
    \pderiv{\alpha_i}{T} &= \pderiv{}{T}\left( \left[1 + \kappa_i\left(1 - \sqrt{\frac{T}{T_{c,i}}}\right)\right]^2 \right) \\
    &= 2 \left[1 + \kappa_i\left(1 - \sqrt{\frac{T}{T_{c,i}}}\right)\right] \cdot \left(-\frac{\kappa_i}{\sqrt{TT_{c,i}}}\right)
\end{align*}
and substituting the first term for the square root of \eqref{eq:alpha} gives
\begin{equation}
    \pderiv{\alpha_i}{T} = -2\kappa_i\sqrt{\frac{\alpha_i}{TT_{c,i}}} \label{eq:dalpha_dT}
\end{equation}
Taking the derivative with respect to temperature a second time
\begin{align*}
    \npderiv{2}{\alpha_i}{T} &= \frac{-2\kappa_i}{\sqrt{T_{c,i}}} \pderiv{}{T}\left(\sqrt{\frac{\alpha_i}{T}}\right) \\
    &= \frac{-2\kappa_i}{\sqrt{T_{c,i}}} \left[\frac{1}{2}\sqrt{\frac{T}{\alpha_i}}\left(\frac{1}{T}\pderiv{\alpha_i}{T} -\frac{\alpha_i}{T^2}\right)\right]
\end{align*}
which can be simplified to
\begin{equation*}
     \npderiv{2}{\alpha_i}{T} = -\kappa_i\sqrt{\frac{\alpha_i}{TT_{c,i}}}\left(\frac{1}{\alpha_i}\pderiv{\alpha_i}{T} - \frac{1}{T} \right)
\end{equation*}
the first term can be substituted using \eqref{eq:dalpha_dT} to give the final formula
\begin{equation}
     \npderiv{2}{\alpha_i}{T} = \frac{1}{2} \pderiv{\alpha_i}{T} \left(\frac{1}{\alpha_i}\pderiv{\alpha_i}{T} - \frac{1}{T} \right)
\end{equation}

\clearpage
\printbibliography
\end{document}
